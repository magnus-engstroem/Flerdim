\documentclass{article}
\usepackage{amsthm}
\usepackage{amssymb}
\usepackage{amsmath}
\usepackage{esdiff}
\usepackage{hyperref}

\setcounter{section}{-1}


\title{Vector Calculus}
\author{Magnus Engstrøm}
\theoremstyle{plain}
\newtheorem{theorem}{Theorem}
\newtheorem{definition}{Definition}
\newtheorem{corollary}{Corollary}

\begin{document}
\maketitle

\tableofcontents


\section{Introduction}

This document was written (loosely) based on the lectures given by 
\href{https://www.ntnu.no/ansatte/mats.ehrnstrom}{Mats Ehrnström}
for the course \href{https://www.ntnu.edu/studies/courses/MA1103#tab=omEmnet}{MA1103 Vector Calculus}
the \href{https://wiki.math.ntnu.no/ma1103/2022v/start}{spring semester of 2022}. 
This document is probably relevant for \href{https://www.ntnu.edu/studies/courses/TMA4105#tab=omEmnet}{TMA4105 Mathematics 2}

\subsection{Notation}

(Ignore this section unless some notation is unclear later)

\begin{itemize}
    \item The notation used for vectors will be both $\vec{x}$ and $\mathbf x$, and we will consider
    only real vectors, thus:
    $$\vec{x}, \mathbf y \in \mathbb R^n \quad \textrm{ for } n \geq 2$$
    $$\vec{x} = \mathbf x$$

    \item $C^m(\mathcal D, \mathcal C)$ is the class of functions $f: \mathcal D \rightarrow \mathcal C$ which are (at least) $m$ times continuously differentiable
    on the domain $\mathcal D$, and with codomain $\mathcal C$

    \item A set $U$ has border $\partial U$. We will consider sets $U \subseteq \mathbb R^k, \quad k \geq 1$.
    A point $\mathbf x \in \partial U$ if 
    $$\forall \epsilon > 0: \quad \{ \mathbf y \in \mathbb R^k; |\mathbf x - \mathbf y| < \epsilon\} \cap U \neq \emptyset \quad \land \quad \{ \mathbf y \in \mathbb R^k; |\mathbf x - \mathbf y| < \epsilon\} \not \subseteq U$$

    If $U = U \cup \partial U$, then $U$ is $closed$, if $\partial U \cap U = \emptyset$ then $U$ is $open$
    \begin{itemize}
        \item Note $\{ \mathbf y \in \mathbb R^k; |\mathbf x - \mathbf y| < \epsilon\}$ is just a $k$-dimentional ball with radius $\epsilon$ and center in $\mathbf x$
    \end{itemize}
\end{itemize}







\section{Geometry}

\subsection{Curves}

\begin{definition}[Parametric curve]
    A parametric curve $\gamma$ is a continuous function
    $$\gamma: I \rightarrow \mathbb R^n$$
    $$ t \mapsto (\gamma_1(t), \gamma_2(t), ..., \gamma_n(t))$$
    where $I \subseteq \mathbb R$ is an interval
    
\end{definition}

\begin{definition}[Smooth curve]
    $\gamma(t)$ is smooth if

    $$|\dot{\gamma}(t)| = \sqrt{(\dot \gamma_1(t))^2 + (\dot \gamma_2(t))^2 + ... + (\dot \gamma_n(t))^2} \neq 0$$
    
\end{definition}

\begin{definition}[Arc length]
    The arc length on $[a, b]$ is the integral
    $$L(\gamma) = \int^{b}_{a} ds = \int^{b}_{a} |\dot{\gamma}(t)| dt $$

    Note the element $ ds :=|\dot{\gamma}(t)| dt$

    
\end{definition}
\subsection{Polar coordinates}


\section{Differentiation}

\subsection{Schwarz' theorem}

\subsection{Jacobi matrix}

\subsection{Hesse matrix}

\subsection{Taylor formulas}

\begin{theorem}[Intermediate value theorem]
    \label{th:ivt}

    Let $F: U \subseteq \mathbb R^n \rightarrow \mathbb R$, $\mathbf x, \mathbf y \in U$
    $\gamma: t \mapsto (1-t) \mathbf x + t \mathbf y$ such that $ t \in [0, 1] \Rightarrow \gamma(t) \in U$

    Then: $\exists \tau \in [0, 1];$

    \begin{equation}\label{eq:ivt}
        F(\mathbf y) - F(\mathbf x) = \nabla F(\gamma(\tau)) \cdot (\mathbf y - \mathbf x)        
    \end{equation}
    
\end{theorem}

\begin{proof}
    Skipped
\end{proof}

In the intermediate value theorem, Theorem \ref{th:ivt} we simply choose two points in the
domain $U$, and draw a line $\gamma$ between them. Then there must
exist a point on that line ($\gamma(\tau)$) at which $F$ has exactly the average slope of the line between 
$F(\mathbf y)$ and $ F(\mathbf x)$.

Rearanging Equation \ref{eq:ivt} gives 
$$F(\mathbf y) = F(\mathbf x) + \nabla F(\mathbf c) \cdot (\mathbf y - \mathbf x), \quad \mathbf c = \gamma(\tau)$$
which we might consider the taylor polynomial of $0^{\textrm{th}}$ degree

\begin{theorem}[Taylor polynomial degree 2]
    \label{th:taylor}
    \begin{equation}\label{eq:taylor}
        F(\mathbf x + \mathbf h) = F(\mathbf x) + \mathbf h \cdot \nabla F(\mathbf x) + \mathbf h^{\rm{T}} [D^2 F(\mathbf x)] \mathbf h + \mathcal O (|\mathbf h|^3)
    \end{equation}

    Note that $[D^2 F(\mathbf x)]$ is a Hesse matrix, such that $\mathbf h^{\rm{T}} [D^2 F(\mathbf x)] \mathbf h$ is a regular matrix quadratic
    form with column vector $\mathbf h$, while $\mathbf h \cdot \nabla F(\mathbf x)$ is the regular dot product between two $\mathbb R^n$-vectors
\end{theorem}

\begin{proof}
    Skipped
\end{proof}
\section{Vector functions}

\subsection{Implicit function theorem}



\subsection{Inverse function theorem}

\begin{theorem}[Inverse function theorem]
    For $\vec F: U \subseteq \mathbb R^n \rightarrow \mathbb R^n$
    where $U$ is open, $\mathbf x \in U$ is fixed and the
    $n \times n$ Jacobi matrix $[ \vec F(\mathbf x)]$ is invertible:

    $$\Rightarrow \exists \tilde{U} \ni \mathbf x; \quad \vec F: \tilde U \rightarrow \vec F(\tilde U)$$

    is invertible.

    Further:
    $$[D (\vec F^{-1})](\mathbf y) = [D \vec F]^{-1} \circ \vec F^{-1}(\mathbf y)$$
    is continuous on its domain $F(\tilde U)$

    Note that $[D \vec F]$ is a Jacobi matrix to be inverted at $ \vec F^{-1}(\mathbf y)$, which
    is a $\mathbb R^n$-vector
    
\end{theorem}

\begin{corollary}[one-domentional inverse function theorem]
    $f: \mathbb R \rightarrow \mathbb R$
    Let $y = f(x)$. If $\diff{f}{x} = f'(x) \neq 0$:

    $$\diff{f^{-1}}{y}(y) = \frac{1}{f'(x)} = \frac{1}{f'(f^{-1}(y))}$$
    
\end{corollary}



\subsection{Lagrange multiplier}

Let:
$$F: U \subseteq \mathbb R^n \rightarrow \mathbb R$$
$$G: U \subseteq \mathbb R^n \rightarrow \mathbb R$$

We want to find the $n$-vector $\mathbf x \in U$ which minimizes $F(\mathbf x)$
while guaranteeing $G(\mathbf x) = 0$

\begin{theorem}[the Lagrange multiplier]
\label{th:lagrangem}

    Any local minimum of $F \in C^1(U, \mathbb R)$ subject to 
    $\{ \mathbf x  \in U; G(\mathbf x) = 0\}$, for a $G \in C^1(U, \mathbb R)$
    on an open $U \subseteq \mathbb R$
    must be where

    \begin{equation}\label{eq:lagrangem}
        \nabla F = \lambda \cdot \nabla G
    \end{equation}
    

    for a real number $\lambda$
    
\end{theorem}

\begin{proof}
    Let the $C^1$ curve $\gamma(t)$ be a parameterization of $\{\mathbf x; G(\mathbf x) = 0 \}$, where the minimum of $F$ along $\gamma(t)$ is $\mathbf x_0 = \gamma(t_0)$.   
    A minimum of $F$ requires:

    $$\diff{}{t} F(\gamma(t)) \Big |_{t = t_0}= 0 = \nabla F(\gamma(t_0)) \cdot \dot{\gamma}(t_0) \quad \textrm{(chain rule)}$$

    Thus the gradient $\nabla F$ is orthogonal to the tangent of 
    every parameterization $\gamma(t)$ of $\{\mathbf x; G(\mathbf x) = 0 \}$
    at the minimum $\mathbf x_0$: $F(\mathbf x_0) \perp \dot{\gamma(t_0)}$

    The minimum is by definition required to satisfy
    $$G(\mathbf x_0) = 0 \Rightarrow \nabla G(\gamma(t_0)) \cdot \dot{\gamma}(t_0) = 0 \quad (\textrm{using the same } \gamma(t))$$
    Which also gives $G(\mathbf x_0) \perp \dot{\gamma(t_0)}$

    Thus:
    $$\nabla F \parallel \nabla G$$
    which we write using a real number $\lambda$:
    $$\nabla F = \lambda \cdot \nabla G$$



\end{proof}




\section{Integrals}

\subsection{Tornelli-Fubini}

\subsection{Variable substitution}

\subsection{Conservative Fields}

\subsection{Greens theorem}

\subsection{Gau{\ss}' theorem}

\subsection{Kelvin-Stokes}


\end{document}


